\documentclass[preprint,11pt,3p]{article}

\usepackage{tocloft}
\usepackage{color}
\usepackage{hyperref}
\usepackage{graphicx}
\usepackage{float}
\usepackage{subcaption}
\usepackage{amsmath} 
\usepackage{tikz} 
\usepackage{epigraph}
\usepackage{lipsum} 
\usepackage{indentfirst}
\usepackage[strings]{underscore}

\usepackage{listings}
\usepackage{color}

\colorlet{light-gray}{gray!10}
\definecolor{javared}{rgb}{0.6,0,0} % for strings
\definecolor{javagreen}{rgb}{0.25,0.5,0.35} % comments
\definecolor{javapurple}{rgb}{0.5,0,0.35} % keywords
\definecolor{javadocblue}{rgb}{0.25,0.35,0.75} % javadoc
\definecolor{main-color}{rgb}{0.6627, 0.7176, 0.7764}
\definecolor{back-color}{rgb}{0.1686, 0.1686, 0.1686}
\definecolor{string-color}{rgb}{0.3333, 0.5254, 0.345}
\definecolor{key-color}{rgb}{0.8, 0.47, 0.196}
\definecolor{asparagus}{rgb}{0.53, 0.66, 0.42}
\definecolor{azure(colorwheel)}{rgb}{0.0, 0.5, 1.0}
\definecolor{ashgrey}{rgb}{0.7, 0.75, 0.71}

\definecolor{shadecolor}{RGB}{150,150,150}

\lstset{
  language=Java,
basicstyle=\small\ttfamily,
keywordstyle=\color{javapurple}\bfseries,
stringstyle=\color{javared},
    keywordstyle = {\color{javapurple}},
    keywordstyle = [2]{\color{asparagus}},
    keywordstyle = [3]{\color{azure(colorwheel)}},
    keywordstyle = [4]{\color{teal}},
    otherkeywords = {:,@@,|,->,>>=,val},
    morekeywords = [2]{;,:,*,@@},
    morekeywords = [3]{->,|},
    morekeywords = [4]{>>=},
commentstyle=\color{javagreen},
morecomment=[s][\color{javadocblue}]{(*}{*)},
numbers=left,
numberstyle=\tiny\color{black},
stepnumber=2,
numbersep=10pt,
tabsize=2,
showspaces=false,
showstringspaces=false,
escapeinside={(*@}{@*)},
% frame=single,
backgroundcolor=\color{light-gray},
frame=lines,
breaklines=true,
postbreak=\mbox{\textcolor{red}{$\hookrightarrow$}\space}}


\renewcommand\epigraphflush{flushright}
\renewcommand\epigraphsize{\normalsize}
\setlength\epigraphwidth{0.7\textwidth}
\renewcommand{\abstractname}{Executive Summary}

\definecolor{titlepagecolor}{cmyk}{1,.60,0,.40}

\DeclareFixedFont{\titlefont}{T1}{ppl}{b}{it}{0.5in}

\makeatletter                       
\def\printauthor{%                  
    {\large \@author}}              
\makeatother
\author{%
    Eric Altenburg \\
    \texttt{ealtenbu@stevens.edu | 10443481}\vspace{20pt} \\
   	Daniel Kimball \\
    \texttt{dkimball@stevens.edu | 10435475}\vspace{20pt} \\
    Hamzah Nizami \\
    \texttt{hnizami1@stevens.edu | 10439133}\vspace{20pt} \\
    Max Shi \\
    \texttt{mshi7@stevens.edu | 10439248}
    }

% The following code is borrowed from: https://tex.stackexchange.com/a/86310/10898

\newcommand\titlepagedecoration{%
\begin{tikzpicture}[remember picture,overlay,shorten >= -10pt]

	\coordinate (aux1) at ([yshift=-15pt]current page.north east);
	\coordinate (aux2) at ([yshift=-410pt]current page.north east);
	\coordinate (aux3) at ([xshift=-4.5cm]current page.north east);
	\coordinate (aux4) at ([yshift=-150pt]current page.north east);

	\begin{scope}[titlepagecolor!40,line width=12pt,rounded corners=12pt]
		\draw
		  (aux1) -- coordinate (a)
		  ++(225:5) --
		  ++(-45:5.1) coordinate (b);
		\draw[shorten <= -10pt]
		  (aux3) --
		  (a) --
		  (aux1);
		\draw[opacity=0.6,titlepagecolor,shorten <= -10pt]
		  (b) --
		  ++(225:2.2) --
		  ++(-45:2.2);
	\end{scope}
	\draw[titlepagecolor,line width=8pt,rounded corners=8pt,shorten <= -10pt]
	  (aux4) --
	  ++(225:0.8) --
	  ++(-45:0.8);
	\begin{scope}[titlepagecolor!70,line width=6pt,rounded corners=8pt]
		\draw[shorten <= -10pt]
		  (aux2) --
		  ++(225:3) coordinate[pos=0.45] (c) --
		  ++(-45:3.1);
		\draw
		  (aux2) --
		  (c) --
		  ++(135:2.5) --
		  ++(45:2.5) --
		  ++(-45:2.5) coordinate[pos=0.3] (d);   
		\draw 
		  (d) -- +(45:1);
	\end{scope}
\end{tikzpicture}%
}

\begin{document}
\begin{titlepage}

\noindent
\titlefont Shifts\par
\epigraph{Technical Implementation Proposal}%
{\textit{CS 554: Web Programming II |  Spring 2021}}
\null\vfill
\vspace*{1cm}
\noindent
\hfill
\begin{minipage}{0.35\linewidth}
    \begin{flushright}
        \printauthor
    \end{flushright}
\end{minipage}
%
\begin{minipage}{0.02\linewidth}
    \rule{1pt}{200pt}
\end{minipage}
\titlepagedecoration
\end{titlepage}




\newpage

\tableofcontents
\newpage

\section{Overview}
With stocks becoming easier for the average person to begin investing in, the idea of shifts is to provide a community where like-minded investors can chat about a certain stock. The list of featured stocks will be updated on a daily basis through web scraping various news outlets and online communities to get the top mentioned stock tickers. Other stock tickers’ chat rooms will be available through a search functionality.

To facilitate discussion, each chatroom for a ticker will have its daily, monthly, 3-month, 1-year, and 5-year charts showing its performance. Other data included would be the general description of the company’s operations, the CEO, where it is located, when it was founded, the market cap, and its average volume. On this page, logged in users will be able to comment under the ticker with their own due diligence, their current positions, or how they think the stock will perform. Additionally, the users will be able to reply to chats from other users to help further allow for a deeper discussion. We believe that hosting an application where users can come together to discuss investing will allow for the topic to reach younger audiences, and help them build a better understanding of financial responsibility.
 

\section{Course Technologies}
\begin{enumerate} 
	\item React - We will use React as the front end framework for this project.
	\item Firebase - We will use Firebase Firestore as a NoSQL database to store chat data and profile information, and Firebase Authentication to authenticate users.
	\item Workers - We will use workers to scrape the internet in real-time to do sentiment analysis on stock tickers.
	\item Socket.io - Socket.io will be used to create real-time chat functionality as well as live updating stock tickers for the user.
\end{enumerate} 

\newpage
\section{Additional Technologies}
\begin{enumerate}
	\item Heroku CI/CD/Hosting - We will use Heroku to enable CI/CD on pushes to the master branch on github and Heroku Dynos to host our application online.
	\item Recharts - Recharts will be used for effective data visualization of stock information.
\end{enumerate}	


\section{Github Repository}
\href{https://github.com/hniz/shifts}{https://github.com/hniz/shifts }

\end{document}





